
\section{Installation and Usage of the Compiler}
\label{sec:using_the_compiler}

This section gives a guide for building and installing the Patmos
tool chain and describes how to use it to compile applications for Patmos. 

% ------------------------------------------------------------------------------
\subsection{Prerequisites}

The Patmos compiler tool chain is developed for the Linux platform. 
In order to build all tools, the following programs and libraries are required:

\begin{description}
\item[cmake, make] \hfill\\
The CMake (at least version 2.8) and \texttt{make} build systems are required to build the
various components of the tool chain, such as the \texttt{clang} compiler or the \texttt{compiler-rt} system library.
\medskip

\item[gcc, g++] \hfill\\
A C and C++ compiler such as GCC is required to build \texttt{clang}, \texttt{llvm}
and \texttt{gold}. It is also possible to use a separate \texttt{clang} installation as 
an alternative to GCC. Compiling with \texttt{clang} will result in shorter compile
times, however, if the compiled application should be debugged with \texttt{gdb}, it is
recommended to use \texttt{gcc} and \texttt{g++}.
\medskip

\item[flex, bison, texinfo] \hfill\\
These tools are required to build tools such as gold successfully.
\medskip

\item[libelf] \hfill\\
The tools in the Patmos tool chain require the development headers of \texttt{libelf},
as this library is used to read and write ELF files.
\medskip

\item[boost] \hfill\\
The simulator requires the program-options library from the \texttt{boost} C++ library
(at least version 1.46).
\medskip

\item[git] \hfill\\
The version control system \texttt{git} is required 
to download the latest development versions of the tools. 
\end{description}

On current Debian based Linux distributions, the following command can be used to install
all the required packages:

\begin{verbatim}
sudo apt-get install git cmake make g++ texinfo flex bison \
  libelf-dev graphviz libboost-dev libboost-program-options-dev
\end{verbatim}

% ------------------------------------------------------------------------------
\subsection{Retrieving and Building the Source Code}

We provide a build script that retrieves, configures and builds all
the required packages automatically (cf. Section~\ref{sec:buildscript}).
Alternatively, the various tool chain source repositories can be acquired individually from 
the \texttt{t-crest} organisation at \texttt{github.com}.\footnote{\url{http://github.com/t-crest/}}, as
described in Section~\ref{sec:manual_build}. 

We created a tag called \texttt{release\_m12} in all tool chain related source repositories mentioned in this report
that marks the current stable release version of the various tools at the time of writing.
To build the released compiler version instead of the latest development version, execute the following command
in all of the source repositories after they have been checked out, and then rerun the build commands.

\begin{verbatim}
git checkout release_m12
\end{verbatim}

Alternatively, \texttt{github.com} provides a link in all of the source repositories to download the source code for that tag as 
zip files.


%\subsubsection{Default Directory Structure}

For the following build instructions we use directory structure presented in Table~\ref{tab:directories}. 
We use \texttt{\textasciitilde/tcrest/} as root directory and \texttt{\textasciitilde/tcrest/local/}
as installation directory in this document. However, the root directory and the installation directory
can be chosen arbitrarily.

\newcommand{\rootdir}[1]{\texttt{\textasciitilde/tcrest/#1}}
% Uuugly!
\newcommand{\myspace}{2mm}
\begin{table}
\centering
\begin{tabularx}{\textwidth}{lX}
Directory & Description \\ \hline
\rootdir{} & Base directory (\texttt{ROOT\_DIR}) \\ [\myspace]
\rootdir{patmos} &  Processor source repository \\ [\myspace]
\rootdir{patmos/simulator} & Simulator sources \\ [\myspace]
\rootdir{patmos/simulator/build/} & Simulator build directory \\ [\myspace]
\rootdir{compiler-rt/} & System library source repository  \\ [\myspace]
\rootdir{compiler-rt/build/} & System library build directory \\ [\myspace]
\rootdir{gold/} & ELF linker and \texttt{binutils} sources \\ [\myspace] 
\rootdir{gold/build/} & Linker and \texttt{binutils} build directory \\ [\myspace]
\rootdir{llvm/} & LLVM source repository \\ [\myspace] 
\rootdir{llvm/tools/clang} & Clang source repository. Must always be a 
			    \texttt{tools/clang} subdirectory of LLVM \\ [\myspace]
\rootdir{llvm/build/} & Build directory for LLVM and clang \\ [\myspace] 
\rootdir{local/} & Install base directory (\texttt{INSTALL\_DIR}) \\ [\myspace]
\rootdir{local/bin/} & Install directory for tool chain binaries. 
                       All binaries in this directory are prefixed 
		       with \texttt{patmos-} by the build script \\ [\myspace]
\rootdir{local/lib/} & Install directory for tool chain libraries \\ [\myspace] 
\rootdir{local/patmos-unknown-unknown-elf/} & Install directory for Patmos 
                                       libraries and headers \\ [\myspace]
\rootdir{misc/} & Support files (e.g., build script, config files) \\ [\myspace]
\rootdir{newlib/} & Sources of standard C library for Patmos \\ [\myspace]
\rootdir{newlib/build/} & Standard library build directory 
\end{tabularx}
\caption{Default directory structure of the tool chain}
\label{tab:directories}
\end{table}

\subsubsection{Building the Tool Chain Using the Build Script}
\label{sec:buildscript}

After installing the prerequisites, download or check out the build script 
from the \texttt{patmos-misc} repository on \texttt{github.com}:

\begin{verbatim}
git clone https://github.com/t-crest/patmos-misc.git misc
\end{verbatim}

To configure the build script, place a configuration file called \texttt{build.cfg}
into the same directory as the \texttt{build.sh} script. 
In this configuration file, set the \texttt{ROOT\_DIR} variable to the directory which should contain all
the repository checkouts and build directories (e.g., \texttt{ROOT\_DIR=\textasciitilde/tcrest/}).

\begin{verbatim}
cd misc
cp build.cfg.dist build.cfg
# edit build.cfg, set ROOT_DIR and INSTALL_DIR
./build.sh
\end{verbatim}

Executing \texttt{build.sh} checks out any missing repository, configures and then builds
all the required tools and libraries of the tool chain. 
By default, the directory structure created by the build script follows the directory structure presented in Table~\ref{tab:directories}. 
All necessary compiler tools are installed to 
\texttt{INSTALL\_DIR/bin}, while Patmos specific libraries such as \texttt{newlib} and 
\texttt{compiler-rt} are installed under \texttt{INSTALL\_DIR/patmos-unknown-unknown-elf/}.
All executables are prefixed with \texttt{patmos-} to differentiate them from non-Patmos related
installations, e.g., \texttt{clang} will be installed as \texttt{patmos-clang}.


\subsubsection{Building the Tool Chain Manually}
\label{sec:manual_build}

To build all tools and libraries manually, check out and build the following repositories as described below.

\paragraph{The Binary Linker \texttt{gold}} \hfill\\

The following commands build the \texttt{gold} linker for Patmos ELF files.
\begin{verbatim}
git clone https://github.com/t-crest/patmos-gold.git gold
mkdir gold/build
cd gold/build
../configure --program-prefix=patmos- --enable-gold=yes \
    --enable-ld=no --enable-plugins --prefix=~/tcrest/local
make all-gold install-gold
\end{verbatim}

%Optionally, the binutils tools \texttt{ar} and \texttt{nm} for creating and inspecting archives 
%can be built with the command
%\begin{verbatim}
%make all-binutils install-binutils
%\end{verbatim}
%They support the \texttt{LLVMgold.so} plugin provided by LLVM, which enables them to work with archive
%files containing bitcode files.

\paragraph{The Compiler Framework \texttt{llvm} and the C Compiler Tool \texttt{clang}} \hfill\\

The \texttt{patmos-clang} repository must be checked out inside the \texttt{patmos-llvm} repository checkout as
\texttt{tools/clang} directory (creating a symlink instead is not supported). 
\texttt{Clang} is then built automatically when \texttt{llvm} is built.

\begin{verbatim}
git clone https://github.com/t-crest/patmos-llvm.git llvm
git clone https://github.com/t-crest/patmos-clang.git \
                                         llvm/tools/clang
mkdir llvm/build
cd llvm/build
cmake -DLLVM_TARGETS_TO_BUILD=Patmos \
      -DCMAKE_BUILD_TYPE=Debug \
      -DCMAKE_INSTALL_PREFIX=~/tcrest/local ..
make all install
\end{verbatim}

% Explanation for building LLVMgold.so?

\paragraph{The C Library \texttt{newlib}} \hfill\\

Newlib uses the \texttt{llvm-ar} and \texttt{llvm-ranlib} tools provided by LLVM to build bitcode 
archives, while \texttt{clang} is used to link files. 

\begin{verbatim}
git clone https://github.com/t-crest/patmos-newlib.git newlib
mkdir newlib/build
cd newlib/build
../configure --target=patmos-unknown-unknown-elf \
    AR_FOR_TARGET=~/tcrest/local/bin/llvm-ar \
    RANLIB_FOR_TARGET=~/tcrest/local/bin/llvm-ranlib \
    LD_FOR_TARGET=~/tcrest/local/bin/clang \
    CC_FOR_TARGET=~/tcrest/local/bin/clang \
    CFLAGS_FOR_TARGET="-target patmos-unknown-unknown-elf -O2" \
    --prefix=~/tcrest/local
make all install
\end{verbatim}

Alternatively, the \texttt{ar} and \texttt{ranlib} tools provided by \texttt{patmos-gold} can 
be used to build \texttt{newlib}, provided that LLVM is configured to build the binutils link time optimisation (LTO) plugin that provides
bitcode file support for the binutils tools.
\paragraph{The Support Library \texttt{compiler-rt}} \hfill \\

The following commands build and install the runtime libraries for Patmos.

\begin{verbatim}
git clone https://github.com/t-crest/patmos-compiler-rt.git \
                                                 compiler-rt
mkdir compiler-rt/build
cd compiler-rt/build
cmake -DCMAKE_TOOLCHAIN_FILE=\
        ../cmake/patmos-clang-toolchain.cmake \
      -DCMAKE_PROGRAM_PATH=~/tcrest/local/bin \
      -DCMAKE_INSTALL_PREFIX=~/tcrest/local ..
make all install
\end{verbatim}

The tool chain file \texttt{patmos-clang-toolchain.cmake} sets up the build environment with the tools of the Patmos tool chain 
that are found in the given program path. The compiler-rt libraries are installed
along with the newlib libraries into the default library lookup path
of the Patmos compiler.

\paragraph{The Patmos Simulator \texttt{pasim}} \hfill \\

The simulator for Patmos was part of Deliverable 2.1 and is part of the Patmos processor repository. Compiling the simulator requires an installation of the 
development headers of libboost and libelf. The following commands build and install the simulator.

\begin{verbatim}
git clone https://github.com/t-crest/patmos.git
mkdir patmos/simulator/build
cd patmos/simulator/build
cmake -DCMAKE_INSTALL_PREFIX=~/tcrest/local ..
\end{verbatim}


For more information on building and using the tools, please refer to the \texttt{README.patmos} files 
that are contained in the source repositories.


% ------------------------------------------------------------------------------
\subsection{Running the Compiler}

%In this section we assume that the tools have been compiled and installed with the provided build script,
%i.e., all tools of the Patmos tool chain are prefixed with \texttt{patmos-}.

The compiler is executed by running the compiler driver tool \texttt{patmos-clang}. This invokes the 
necessary tools to compile, assemble and link an application. 

In order to build an application for Patmos, the target platform triple, the output file name and the source files (either C code or bitcode files) 
need to be provided to \texttt{patmos-clang}.

\begin{verbatim}
patmos-clang -target patmos-unknown-unknown-elf -o hello main.c hello.c
\end{verbatim}

This will compile the source files to bitcode files, link in \texttt{newlib} and the system libraries,
and execute \texttt{patmos-gold} to create the final executable application. If the option \texttt{-v} is passed
to \texttt{patmos-clang}, the tools print out all commands that are executed in the background, as well as additional information
about search paths and linked libraries.

After compilation, the application is executed in the simulator by using

\begin{verbatim}
pasim hello
\end{verbatim}

\subsubsection{Driver Options}

The \texttt{patmos-clang} driver can also be used to generate bitcode files, to link bitcode files, or to 
emit assembler code. The driver supports the following modes of operation (for brevity, we omit the
\texttt{-target patmos-unknown-unknown-elf} argument):


\begin{description}
\item[\texttt{patmos-clang -c <inputs>}] \hfill\\
  \begin{tabular}{ll}
  Input:   & \texttt{.c} C source file \\
  Output:  & \texttt{.o} or \texttt{.bc} bitcode files \\
  Actions: & compile each input file to a bitcode file
  \end{tabular}

\item[\texttt{patmos-clang -S <inputs>}] \hfill\\
  \begin{tabular}{ll}
  Input:   & \texttt{.c} C source file \\
  Output:  & \texttt{.s} or \texttt{.ll} human readable bitcode files \\
  Actions: & compile each input file to a human readable bitcode file
  \end{tabular}

\item[\texttt{patmos-clang -emit-llvm <inputs>}] \hfill\\
  \begin{tabular}{ll}
  Input:   & \texttt{.c} C source file, \texttt{.bc} bitcode object file, \texttt{.a} bitcode files archive \\
  Output:  & bitcode file \\
  Actions: & compile to bitcode, link all input files, link with standard libraries and start code \\
\end{tabular}

\item[\texttt{patmos-clang -fpatmos-emit-obj -c <inputs>}] \hfill\\
  \begin{tabular}{ll}
  Input:   & \texttt{.c} C source file \\
  Output:  & \texttt{.o} Patmos relocatable ELF \\
  Actions: & compile each input file to a Patmos relocatable ELF file
  \end{tabular}

\item[\texttt{patmos-clang -fpatmos-emit-obj -S <inputs>}] \hfill\\
  \begin{tabular}{ll}
  Input:   & \texttt{.c} C source file \\
  Output:  & \texttt{.s} Patmos assembly file \\
  Actions: & compile each input file to a Patmos assembly file
  \end{tabular}

\item[\texttt{patmos-clang -fpatmos-emit-obj <inputs>}] \hfill\\
  \begin{tabular}{ll}
  Input:   & \texttt{.c} C source file, \texttt{.bc} bitcode object file, \texttt{.a} bitcode files archive \\
  Output:  & \texttt{.o} Patmos relocatable ELF \\
  Actions: & compile to bitcode, link all input files, link with standard libraries and start code, \\
	   & compile to relocatable ELF
\end{tabular}

\item[\texttt{patmos-clang -fpatmos-emit-asm <inputs>}] \hfill\\
  \begin{tabular}{ll}
  Input:   & \texttt{.c} C source file, \texttt{.bc} bitcode object file, \texttt{.a} bitcode files archive \\
  Output:  & \texttt{.o} Patmos assembly file \\
  Actions: & compile to bitcode, link all input files, link with standard libraries and start code, \\
	   & compile to Patmos assembly
\end{tabular}

\item[\texttt{patmos-clang -o <output> <inputs>}] \hfill\\
  \begin{tabular}{ll}
  Input:   & \texttt{.c} C source file, \texttt{.bc} bitcode object file, \texttt{.a} bitcode files archive \\
  Output:  & Patmos executable ELF \\
  Actions: & compile to bitcode, link with standard libraries and start code, \\
           & compile to relocatable ELF, create Patmos executable ELF
  \end{tabular}

\end{description}

The compiler accepts standard options such as \texttt{-I}, \texttt{-L} and \texttt{-l} to define additional
lookup paths for header files and libraries and to link with (static) libraries.
The behaviour of the linker can be controlled with additional options for \texttt{patmos-clang} 
as shown in Table~\ref{tab:linker_options}.

\begin{table}
\centering
\begin{tabular}{ll}
Option & Description \\ \hline
\texttt{-mfloat-abi=none} & Do not use software floating point libraries when linking \\
\texttt{-nostdlib} & Do not use standard libraries such as \texttt{libc} when linking \\
\texttt{-nolibc} & Do not use \texttt{libc} when linking \\
\texttt{-nodefaultlibs} & Do not use platform system libraries when linking\\
\texttt{-nostartfiles} & Do not use the \texttt{crt0} start file when linking \\
\texttt{-nolibsyms} & Do not use symbol definition files for runtime libraries when\\
                    & linking. Those files prevent the linker from removing any \\
		    & functions for which calls might be generated by the compiler \\
		    & backend, such as software division or \texttt{memcpy} \\
\texttt{-fpatmos-link-object} & Link as object, i.e., do not link in any libraries or start code 
\end{tabular}
\caption{Options for \texttt{patmos-clang} that control the default behaviour of the linker}
\label{tab:linker_options}
\end{table}

\subsubsection{Libraries}

The Patmos tool chain supports static libraries. Libraries are archives that contain
bitcode files. The archives are created by using either the \texttt{ar}
tool provided by the host system or by using \texttt{patmos-ar} from the \texttt{patmos-gold} binutils. The tool
\texttt{patmos-llvm-nm} can be used to inspect the content of the archive.

\begin{verbatim}
ar q libtest.a *.bc
# show the contents of libtest.a
patmos-llvm-nm libtest.a
# compile and link with the created library
patmos-clang -target patmos-unknown-unknown-elf -o app main.c -ltest
\end{verbatim}


% Anything else? -Wl, .. 

