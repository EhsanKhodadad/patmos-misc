\section{Building a Hello World Application}

In this section we show how to compile and simulate a simple Hello World application with the Patmos tool chain, and present the output of the
various tools.

The program we use in this example is
\footnotesize
\begin{verbatim}
$ cat hello.c
\end{verbatim}
\verbatiminput{data/hello.c}

\normalsize
The following command compiles the file \texttt{hello.c} into bitcode, using a human readable output format.
Note that the \code{-target} option is optional for the default target triple, if the compiler driver program is called as
\code{patmos-clang}.
\footnotesize
\begin{verbatim}
$ patmos-clang -target patmos-unknown-unknown-elf -o hello.ll -S hello.c
$ cat hello.ll
\end{verbatim}
\verbatiminput{data/hello.ll}

\normalsize
The following command compiles the file \texttt{hello.c} into Patmos assembly code:
\footnotesize
\begin{verbatim}
$ patmos-clang -target patmos-unknown-unknown-elf -o hello.s \
                                -fpatmos-emit-asm -S hello.c
$ cat hello.s
\end{verbatim}
\verbatiminput{data/hello.s}

\normalsize
We now show how to link the previously generated bitcode file with the standard libraries and start code and compile it to a Patmos ELF
executable. The \texttt{-target} option can be omitted here since the target information is already stored in the bitcode file. 
The executable is then simulated using the \texttt{pasim} simulator.
\footnotesize
\begin{verbatim}
$ patmos-clang -o hello hello.ll
$ pasim -M fifo -m 64k hello
\end{verbatim}
\verbatiminput{data/hello.out}

\normalsize

